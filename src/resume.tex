\documentclass[]{friggeri-cv}

\usepackage[portuguese]{babel}

\begin{document}

\header{João}{Madureira}{Software Developer}

\begin{aside}
\section{contact}
Some street somewhere
There's this brown door
That has this number
Portugal
~
+351 555 555 55
~
\href{mailto:myemail@gmail.com}{myemail@gmail.com}
\href{https://github.com/jmadureira}{github.com/jmadureira}
\section{languages}
portuguese native
english proficiency
\end{aside}

%----------------------------------------------------------------------------------------
%       WORK EXPERIENCE SECTION
%----------------------------------------------------------------------------------------

\section{work experience}

\begin{entrylist}
%------------------------------------------------
\entry
{12/2012-Now}
{Blip Web Designers}
{Porto, Portugal}
{\emph{Java Developer} \\
Development and maintenance of Betfair's Sportsbook website which included:
\begin{itemize}
  \item Both international ( \href{https://www.betfair.com/sport}{https://www.betfair.com/sport} ) and Italian ( \href{https://www.betfair.it/sport}{https://www.betfair.it/sport} ) websites;
  \item Development and support for some of Betfair's signature products like \emph{Cash Out Multiples} and \emph{Price Rush};
  \item High performance and scalable web services supporting all of Betfair's sites and mobile applications.
\end{itemize}
Technologies and tools used:
\begin{itemize}
  \item Spring, Tomcat, Jetty for the website and services;
  \item Couchbase for cashing purposes;
  \item ElasticSearch for search and indexing purposes;
  \item Maven, Jenkins and Chef for build and deployment management;
  \item TestNG and Selenium for integration tests.
\end{itemize}
}
%------------------------------------------------
\entry
{05/2011-11/2012}
{Portugal Telecom}
{Lisbon, Portugal}
{\emph{Software Consultant} \\
Development of an application monitoring solution for Portugal Telecom. Worked on UI components developed with:
\begin{itemize}
\item Ruby on Rails, JQuery, Sinatra for the website;
\item memcached for caching purposes;
\item Apache ActiveMQ and Camel for messaging and routing;
\item MySQL for the datasource.
\end{itemize}
}
%------------------------------------------------
\entry
{11/2010-04/2011}
{Nokia Siemens Networks}
{Aveiro, Portugal}
{\emph{Software Developer} \\
Design and implementation of a functional prototype using several new concepts that could be added to NSN's main network and performance management solution. This analysis not included:
\begin{itemize}
\item Oracle and PostgreSQL as database solutions;
\item Portlets and GWT as web components;
\item EMF as a modeling tool.
\end{itemize}
}
%------------------------------------------------
\entry
{09/2009-10/2010}
{Nokia Siemens Networks}
{Aveiro, Portugal}
{\emph{Software Developer} \\
Migration of the continuous integration process used by NSN's network and
performance management solution to a centralized (cloud based) environment.
This new environment used:
\begin{itemize}
\item Bamboo for the frontend interface;
\item Ruby on Rails for virtual machine and artifact management;
\item SubVersion and Maven for source control and build management.
\end{itemize}
}
\end{entrylist}
\begin{entrylist}
\entry
{12/2007-09/2009}
{Nokia Siemens Networks}
{Aveiro, Portugal}
{\emph{Software Developer} \\
Prototyping and development of an ETL solution to be used by NSN's new network and performance management solution base on ESB (Enterprise Service Bus). \\
Implementation of a production pipe using:
\begin{itemize}
\item Ant for build management, later migrated to Maven;
\item Clearcase for source control management, later migrated to SVN;
\item CruiseControl for the frontend interface and build management purposes;
\item Virtualization of test machines using KVM.
\end{itemize}
}
%------------------------------------------------
\entry
{09/2006-11/2007}
{Siemens Networks}
{Aveiro, Portugal}
{\emph{Software Developer Trainee} \\
Activity developed as part of a professional trainee program.\\
Design and specification of a new data format used by Siemens to describe a
telecommunication network. This data format was meant to be less verbose and
error-prone. Development and test of modules for Siemens' performance
management solution in both C++ and Java.
}
%------------------------------------------------
\entry
{03/2006-8/2006}
{Siemens SA}
{Perafita, Portugal}
{\emph{Software Developer Trainee} \\
Activity developed as part of a curricular trainee program.\\
Development of an IDE based on Eclipse RCP to aid on the development of
Siemens’s network performance management solution.
}
\end{entrylist}

%----------------------------------------------------------------------------------------
%       TRAINING SECTION
%----------------------------------------------------------------------------------------

\section{training}

\begin{entrylist}
%------------------------------------------------
\entry
{05/2014}
{Scrum Aliance}
{Porto, Portugal}
{Scrum Master course and CSM (Certified Scrum Master) exam.}
%------------------------------------------------
\entry
{03/2012}
{VMWare Educational Services}
{Lisbon, Portugal}
{Core Spring.}
%------------------------------------------------
\entry
{02/2008}
{Quality Tree Software}
{Aveiro, Portugal}
{Agile Testing and TDD (Test Driven Development).}
%------------------------------------------------
\entry
{09/2007}
{Learning Tree International}
{Aveiro, Portugal}
{Java Enterprise Application Development - JSP, JSF, EJB3, JPA, JMS.}
\entry
{08/2007}
{Learning Tree International}
{Aveiro, Portugal}
{Advanced Java Programming - Annotations, concurrency, class loaders, NIO, JMX}

\end{entrylist}

%----------------------------------------------------------------------------------------
%       EDUCATION SECTION
%----------------------------------------------------------------------------------------

\section{education}

\begin{entrylist}
%------------------------------------------------
\entry
{09/2005-07/2006}
{British Council}
{Porto, Portugal}
{CPE (Certificate of Proficiency in English).\\
ESOL exam completed with a level B (between 75\% and 79\%.
}
%------------------------------------------------
\entry
{09/2001-09/2006}
{Faculty of Engineering from the University of Porto}
{Porto, Portugal}
{Licentiate in Informatics and Computing Engineering completed with an average of 16 out of 20.
}

\end{entrylist}

\end{document}
