\documentclass{res}
\usepackage[utf8]{inputenc}
\usepackage[portuguese]{babel}
\usepackage{eurosym}
\usepackage{graphics}
\usepackage{hyperref}
%%\usepackage{floatflt}
\usepackage{graphics}
\usepackage{graphicx}
\usepackage{color}

\begin{document}

\name{João Paulo Pinto Soares Madureira\\[12pt]}     % the \\[12pt] adds a blank
                               				         % line after name
                               				         
\address{\bf  PRESENT ADDRESS\\Places and stuff\\cities and the like\\(351) 55
555 555} \address{\bf PERMANENT ADDRESS\\Home town\\ home city\\  (351) 55 555
5555}
                               				               


\begin{resume}

\section{WORK EXPERIENCE}

\vspace{-0.1in}
\begin{tabbing}
\hspace{2.3in}\= \hspace{2.6in}\= \kill % set up two tab positions
{\bf Software Consultant} \>Portugal Telecom (PT)     \>05/2011 - Present\\
\>Lisbon, Portugal
\end{tabbing}\vspace{-20pt}      % suppress blank line after tabbing
Development of an application monitoring solution for Portugal Telecom.
Worked on UI components developed with Ruby on Rails, JQuery, Sinatra and memcached, backend message routing with Apache ActiveMQ and Camel, and MySQL for the datasource.

\begin{tabbing}
\hspace{2.3in}\= \hspace{2.6in}\= \kill % set up two tab positions
{\bf Software Developer} \>Nokia Siemens Networks (NSN)    
\>11/2010 - 04/2011\\ \>Aveiro, Portugal
\end{tabbing}\vspace{-20pt}      % suppress blank line after tabbing
Design and implementation of a functional prototype using several new concepts that could be added to NSN's main network and performance management solution.
This analysis not only included databases (ORACLE e PostgreSQL) but also web components (Portlets, J2EE and GWT) and modeling tools (EMF).

\begin{tabbing}
\hspace{2.3in}\= \hspace{2.6in}\= \kill % set up two tab positions
{\bf Software Developer} \>Nokia Siemens Networks (NSN)    
\>02/2010 - 10/2010\\ \>Aveiro, Portugal
\end{tabbing}\vspace{-20pt}      % suppress blank line after tabbing
Migration of the continuous integration process used by NSN's network and performance management solution to a centralized (cloud based) environment.
This new environment used Bamboo for the frontend interface, Ruby on Rails for virtual machine and artifact management, SubVersion for source control and Maven for build management.

\begin{tabbing}
\hspace{2.3in}\= \hspace{2.6in}\= \kill % set up two tab positions
{\bf Software Developer} \>Nokia Siemens Networks (NSN)    
\>01/2009 - 09/2009\\ \>Aveiro, Portugal
\end{tabbing}\vspace{-20pt}      % suppress blank line after tabbing
Design and implementation of a prototype for ETL handling based on ESB (Enterprise Service BUS).
The purpose of this prototype was to use Apache ServiceMix to extracts performance metrics from network elements, process them and load into NSN's network and performance management solution.
Development of a continuous integration process using CruiseControl for the frontend interface, SubVersion for source control, Maven for build management and virtualization of test machines through KVM.

\begin{tabbing}
\hspace{2.3in}\= \hspace{2.6in}\= \kill % set up two tab positions
{\bf Software Developer} \>Nokia Siemens Networks (NSN)    
\>12/2007 - 12/2008\\ \>Aveiro, Portugal
\end{tabbing}\vspace{-20pt}      % suppress blank line after tabbing
Development of an ETL mechanism in Java to be used NSN's new network and performance management solution.
Implementation of a production pipe using Ant, Clearcase and CruiseControl for this new solution.

\begin{tabbing}
\hspace{2.3in}\= \hspace{2.6in}\= \kill % set up two tab positions
{\bf Software Developer} \>Siemens Networks    
\>09/2006 - 11/2007\\ \>Aveiro, Portugal
\end{tabbing}\vspace{-20pt}      % suppress blank line after tabbing
Activity developed as part of a professional trainee program.
Design and specification of a new data format used by Siemens to describe a telecommunication network.
This data format was meant to be less verbose and error-prone.
Development and test of modules for Siemens' performance management solution in both C++ and Java.

\begin{tabbing}
\hspace{2.3in}\= \hspace{2.6in}\= \kill % set up two tab positions
{\bf Software Developer} \>Siemens SA    
\>03/2006 - 8/2006\\ \>Perafita, Portugal
\end{tabbing}\vspace{-20pt}      % suppress blank line after tabbing
Activity developed as part of a curricular trainee program.
Development of an IDE based on Eclipse RCP to aid on the development of Siemens’s network performance management solution.

\section{TRAINING}

\begin{tabbing}
\hspace{2.3in}\= \hspace{2.6in}\= \kill % set up two tab positions
{\bf } \>Quality Tree Software    
\>02/2008\\ \>
\end{tabbing}\vspace{-20pt}
Agile Testing and TDD (Test Driven Development).

\begin{tabbing}
\hspace{2.3in}\= \hspace{2.6in}\= \kill % set up two tab positions
{\bf } \>Learning Tree International    
\>09/2007\\ \>
\end{tabbing}\vspace{-20pt}
Java Enterprise Application Development - JSP, JSF, EJB3, JPA, JMS.

\begin{tabbing}
\hspace{2.3in}\= \hspace{2.6in}\= \kill % set up two tab positions
{\bf } \>Learning Tree International    
\>08/2007\\ \>
\end{tabbing}\vspace{-20pt}
Advanced Java Programming - Annotations, concurrency, class loaders, NIO, JMX

\section{EDUCATION}

\begin{tabbing}
\hspace{2.3in}\= \hspace{2.6in}\= \kill % set up two tab positions
{\bf } \>British Council
\>09/2005 - 07/2006\\ \>Porto, Portugal
\end{tabbing}\vspace{-20pt}
CPE (Certificate of Proficiency in English).
English course and ESOL exam completed with a level B (between 75\% and 79\%).

\begin{tabbing}
\hspace{2.3in}\= \hspace{2.6in}\= \kill % set up two tab positions
{\bf } \>Faculdade de Engenharia
\>09/2001 - 09/2006\\ \>University of Porto, Portugal
\end{tabbing}\vspace{-20pt}
Licentiate in Informatics and Computing Engineering.
Completed with an average of 16 out of 20.

\end{resume}

\end{document}
